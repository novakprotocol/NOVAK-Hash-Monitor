\documentclass[12pt]{article}
\usepackage{amsmath, amssymb, amsthm}
\usepackage{fullpage}
\usepackage{mathrsfs}
\usepackage{hyperref}
\usepackage{graphicx}
\usepackage{enumitem}

\title{\textbf{NOVAK Protocol — Formal Mathematical Appendix}\\
\large Deterministic Execution Integrity, HVET/EIR/RGAC, and Safety Gate Model}
\author{Matthew S. Novak (2025)}
\date{}

\begin{document}
\maketitle

\tableofcontents
\newpage

% ---------------------------------------------------------
\section{Introduction}
% ---------------------------------------------------------

This appendix provides the complete formal mathematical foundation of the
NOVAK Protocol: a deterministic proof-before-action execution–integrity system
governing autonomous computation, regulatory logic evaluation, and machine-level decision handling.

The appendix defines:

\begin{itemize}
  \item Deterministic execution mapping
  \item Cryptographic binding equations (HVET)
  \item Execution Identity Receipt (EIR)
  \item Recursive Global Audit Chain (RGAC)
  \item Safety Gate functions (logical, physical, and psycho-social)
  \item Drift-space, ambiguity-space, and deception-space models
  \item Constraint sets defining legal, regulatory, and interpretive invariants
\end{itemize}

This appendix is normative and mathematically authoritative.

% ---------------------------------------------------------
\section{Global Definitions}
% ---------------------------------------------------------

Let:

\begin{itemize}
  \item $\mathcal{D}$ : The attested input domain  
  \item $\mathcal{R}$ : The governing rule domain  
  \item $\mathcal{O}$ : The output domain  
  \item $\mathcal{T}$ : The timestamp domain  
  \item $\mathcal{I}$ : The execution-identity domain  
  \item $H(\cdot)$ : A cryptographic hash function (e.g., SHA-256)  
  \item $\oplus$ : Concatenation operator  
  \item $\mathcal{C}$ : RGAC chain  
  \item $\mathcal{S}$ : Safety Gate state  
\end{itemize}

NOVAK defines a deterministic tri-mapping:

\[
f : (\mathcal{D},\mathcal{R}) \rightarrow \mathcal{O}
\]

subject to purity constraints.

% ---------------------------------------------------------
\section{Equation 1 — Deterministic Purity Law}
% ---------------------------------------------------------

For any rule set $R \in \mathcal{R}$ and input $D \in \mathcal{D}$:

\[
f(D,R) = O
\quad\text{and}\quad
f(D,R) = O' \implies O = O'
\]

No stochastic behavior is permitted.

% ---------------------------------------------------------
\section{Equation 2 — Rule Non-Malleability}
% ---------------------------------------------------------

A rule $R$ is non-malleable if:

\[
H(R) = H(R') \implies R = R'
\]

Hence, rule identity is hash-equivalent to rule content.

% ---------------------------------------------------------
\section{Equation 3 — Input Non-Malleability}
% ---------------------------------------------------------

\[
H(D) = H(D') \implies D = D'
\]

No silent alteration of inputs may occur.

% ---------------------------------------------------------
\section{Equation 4 — Output Non-Malleability}
% ---------------------------------------------------------

\[
H(O) = H(O') \implies O = O'
\]

Together, Eq. 2–4 form the “Immutable Triplet.”

% ---------------------------------------------------------
\section{Equation 5 — Hash-Verified Execution Trace (HVET)}
% ---------------------------------------------------------

The HVET value is defined as:

\[
HVET = H\big( H(R) \oplus H(D) \oplus H(O) \oplus T \big)
\]

This binds the entire computation into one irreversible token.

% ---------------------------------------------------------
\section{Equation 6 — Execution Identity Receipt (EIR)}
% ---------------------------------------------------------

An EIR is a tuple:

\[
EIR = (ID, H(R), H(D), H(O), T, HVET)
\]

where:

\[
ID = H(\mathcal{I})
\]

is the hashed execution identity (machine, user, model, agent).

% ---------------------------------------------------------
\section{Equation 7 — RGAC Linking Function}
% ---------------------------------------------------------

Let $\mathcal{C} = (EIR_1, EIR_2, \dots, EIR_n)$.

A chain entry is linked as:

\[
Link_i = H\big(HVET_i \oplus HVET_{i-1}\big)
\]

with $HVET_0 = GENESIS$.

% ---------------------------------------------------------
\section{Equation 8 — Full RGAC Entry}
% ---------------------------------------------------------

Each RGAC entry is:

\[
RGAC_i = \big(EIR_i, Link_i, Link_{i-1}\big)
\]

forming a recursive lineage of computation.

% ---------------------------------------------------------
\section{Equation 9 — Safety Gate Function}
% ---------------------------------------------------------

Let $S$ be the safety-gate decision function:

\[
S(D,R,O) =
  \begin{cases}
    1 & \text{if valid and no violation detected} \\
    0 & \text{otherwise}
  \end{cases}
\]

The Safety Gate governs execution permission.

% ---------------------------------------------------------
\section{Equation 10 — PL-X Physical Drift Bound}
% ---------------------------------------------------------

Let $\Delta_{phys}$ represent physical-layer drift (timing, voltage, metastability):

\[
\Delta_{phys} \leq \epsilon_{phys}
\]

Execution is permitted iff:

\[
\epsilon_{phys} < \epsilon_{threshold}
\]

% ---------------------------------------------------------
\section{Equation 11 — PS-X Psycho-Social Manipulation Bound}
% ---------------------------------------------------------

Let $\Delta_{ps}$ represent deceptive or adversarial manipulations:

\[
\Delta_{ps} \leq \epsilon_{ps}
\]

Violation:

\[
\Delta_{ps} > \epsilon_{ps} \implies S(D,R,O) = 0
\]

% ---------------------------------------------------------
\section{Equation 12 — Composite High-Assurance Integrity Constraint}
% ---------------------------------------------------------

NOVAK requires:

\[
S(D,R,O) = 1 \iff
\begin{cases}
\Delta_{phys} \leq \epsilon_{phys} \\
\Delta_{ps} \leq \epsilon_{ps} \\
HVET \text{ verified} \\
RGAC \text{ recursion valid}
\end{cases}
\]

% ---------------------------------------------------------
\section{Equation 13 — Proof-Before-Action Final Law}
% ---------------------------------------------------------

Execution may only occur if:

\[
Execute(D,R) \iff S(D,R,f(D,R)) = 1
\]

Else:

\[
Execute(D,R) = \varnothing
\]

This is the NOVAK foundational enforcement rule.

% ---------------------------------------------------------
\section{Drift-Space Formalization}
% ---------------------------------------------------------

Define a drift space:

\[
\mathscr{D} = \{ \delta : D \rightarrow D' \mid H(D) \neq H(D') \}
\]

NOVAK prohibits undetected movement through $\mathscr{D}$.

% ---------------------------------------------------------
\section{Ambiguity-Space Formalization}
% ---------------------------------------------------------

Define ambiguity space:

\[
\mathscr{A} = \{ D \mid D \text{ interpreted differently by } R_1, R_2 \}
\]

NOVAK requires:

\[
|\mathscr{A}| = 0
\]

after rule normalization.

% ---------------------------------------------------------
\section{Deception-Space Formalization}
% ---------------------------------------------------------

\[
\mathscr{S}_{adv} = \{ (D,R) \rightarrow O' \mid O' \neq f(D,R) \}
\]

NOVAK enforces:

\[
\forall O' \in \mathscr{S}_{adv},\; S(D,R,O') = 0
\]

% ---------------------------------------------------------
\section{Conclusion}
% ---------------------------------------------------------

These equations formalize the mathematical invariants underlying the NOVAK Protocol’s deterministic execution, cryptographic binding, and cross-domain safety guarantees.

\end{document}
